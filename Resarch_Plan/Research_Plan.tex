\documentclass[11pt,a4paper]{article}
\pdfoutput=1

%=========================Package begin========================%
\usepackage{amsmath}
\usepackage{color}
\usepackage{multicol}
\usepackage{CJK}
\usepackage{url}
\usepackage{geometry}
\usepackage{caption}
\usepackage{indentfirst}
\usepackage{setspace}           % set line to line space 
%=========================Package end========================%
\geometry{left=2.5cm,right=2.5cm,top=2.5cm, bottom=2.5cm}

%=================================================================%
\begin{document}

\begin{spacing}{1.25}           % set line to line space is 1.25

\begin{center}
\section*{\huge Research Plan}
\end{center}

\section*{Introduction of my background}    % Part One
My name is Wei Zhang. I am major in Particle Physics and Nuclear Physics at Central China Normal University (CCNU) from September 1, 2017 to now. And I will go to the Southern Methodist University (SMU), Dallas, Texas, as an exchange visiting scholar for half a year.
\section*{My field of research}    % Part two 
Physical Electronics relates to board-level and chip-level electronic designs to perform signal-shaping, massive and high-speed data acquisition, data transmission, and data storage, which is widely demanded and used in various situations in our daily life or scientific experiments. For example, the high-speed Ethernet Switch, data acquisition system for different systems in industrial productions, scientific experiments, medical equipment, and consumer electronics, etc.

To fulfill the demands mentioned above, the Printed Circuit Board (PCB), the topological structure and Application Specific Integrated Circuit (ASIC) of electrical system should be elaborately designed to tackle with high-speed signals and large-scale data channels. These two aspects constitute the main contents about my research direction.

\section*{Advisor's information}     % Part three
\begin{multicols}{2}
\textbf{Name:} Jingbo Ye   

\textbf{Title:} Professor 

\textbf{Phone:} (214)768-2114

\textbf{Cell Phone:} (214)783-0054

\textbf{College:} Department of Physics

\textbf{Institution:} Southern Methodist University (SMU)

\textbf{Address:} Dallas, TX 75275 
 
\textbf{Email:} \texttt{yejb@physics.smu.edu} 

\end{multicols}

\section*{Advisor's Research in SMU}     % Part four 
Jingbo Ye, the professor sponsoring and inviting me as visiting scholar, will be my research advisor in SMU. He is a Physics Research Professor for SMU's Data Link Group in Department of Physics, collaborates with many institutions such as CERN, is in the lead of this research in the high energy physics community worldwide. The optical link system that are being developed, based on the Gigabit Transceiver (GBT) and LOC ASIC families for the phase-I upgrade program of the LHC experiments, including ATLAS, are the state-of-the-art in the field. Optical links for phase-I will operate at 5 to 10 times faster than the current systems in the LHC experiments. For the LHC phase-II upgrade, SMU and CERN, together with other institutions, form the lpGBT (low power GBT) common project and are developing ASICs for optical links that will double the speed of those in phase-I.

\section*{My research plan in SMU}     % Part four 
\end{spacing}
\end{document}
